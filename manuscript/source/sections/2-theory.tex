\documentclass[../main.tex]{subfiles}

\begin{document}
\section{discuss healthy immigrant and diversity}
\subsection{Liu, 2006, Higher}
Infant mortality is an important marker of a community’s health and social well-being. Main risk factors associated with a high infant mortality rate include lower socio-economic status, lack of prenatal care, less education, tobacco and substance abuse during pregnancy, being a teenaged, black woman, and poor health of the mother (1–7). Several studies have shown that foreign-born women as a group have better pregnancy outcomes, such as lower rate of low birth weight infants, less preterm deliveries and lower infant mortality rate (IMR) compared to 1Office of Family Health, New York City Department of Health and Mental Hygiene, New York. 2Present address: Bureau of HIV/AIDS, New York City Department of Health and Mental Hygiene, New York. 3Present address: Bureau of Tuberculosis Control, New York City Department of Health and Mental Hygiene, New York. 4Correspondence should be directed to Kai-Lih Liu, Ph.D., MPH, HIV Epidemiology Program, New York City Department of Health and Mental Hygiene, 346 Broadway, Room 701, Box 44, New York 10013; e-mail: kliu@health.nyc.gov. women who were born in the United States (US) (812). These better outcomes have been explained in part by the phenomenon of positive selectivity called healthy immigrant effect, whereby immigrants who were able to come to the US are healthier than their fellow countrymen who stayed in their country of birth (10). However, findings from several studies of perinatal and infant health among various immigrant groups display a far more complicated picture, as their health outcomes tend to vary when foreignborn groups are stratified by race/ethnicity and by specific country of maternal birth (13–17). For example, adverse pregnancy outcomes have been found to be more frequent among foreign-born black (African or Caribbean) women than among white women or US-born black women (18, 19).

\section{discuss why preventable hospitalization}

\subsection{Ratakounda}


info for potentialy preventable hospitalization:

otentially preventable hospitalizations (PPHs)dalso known as ambulatory care-sensitive conditions or prevention quality indicatorsdcan often be avoided if timely and effective outpatient care is provided.1 PPHs are an expensive and largely inefficient use of health care services.2-4 Moreover, PPHs are not value-added and may increase harm to vulnerable patient populations such as those with sensory disabilities.5 In this study, we aimed to examine the risk of hearing, vision, and dual sensory loss for PPH.

\section{discuss all}

\subsection{Bloodworth theseis chapter 4}

Chronic Conditions and Primary Care Chronic diseases are the leading cause of morbidity and mortality in the United States and can result in high levels of healthcare expenditures. In 2010, chronic diseases accounted for 70\% of causes of death in the United States (CDC, 2014), and in 2012, approximately 50\% of adults (117 million people) had at least one chronic disease (Ward, Schiller, & Goodman, 2014). Chronic disease related illnesses made up 86\% of all healthcare expenditures in 2010 (Gerteis et al., 2014), and healthcare expenditures accounted for 17.4\% of Gross Domestic Product spending in the United States in 2013 (Hartman, Martin, Lassman, & Catlin, 2015). Access to primary care and preventive services can potentially help curb the cost associated with chronic disease (CDC, 2013). Early screening and detection can prevent chronic diseases from worsening and sometimes from occurring all together, which could help curb the costs associated with chronic conditions in some cases (CDC, 2013). It is estimated that if more people utilized preventive services, two million life years could be saved annually (Maciosek, Coffield, Flottemesch, Edwards, & Solberg, 2010). However, in the US, only about 3 cents of every dollar are spent on prevention (Kemp et al., 2012). It is well established in the literature that patient cost-sharing, such as copay, can affect the utilization of preventive services (Solanki & Schauffler, 1999). Many people do not have timely access to or cannot afford necessary preventive car

81 (Cheung, Wiler, Lowe, & Ginde, 2012; Meissner, Klabunde, Breen, & Zapka, 2012), even if a service is covered under an insurance plan. To overcome this access barrier, the ACA stipulates that all private insurance plans purchased beginning in 2010 must cover all USPSTF A and B level preventive services at no cost-sharing to the patient. Studies have shown that among the privately insured, this preventive service coverage mandate had a positive effect on preventive services recommended on an annual basis (such as blood pressure check and flu shot) (Han, Robin Yabroff, Guy, Zheng, & Jemal, 2015), but had limited effect on cancer screenings, which are not recommended on an annual basis (Mehta et al., 2015). Prevention Quality Indicators: Measuring Primary Care Effectiveness Getting people covered is just the first step. Even with 100\% insurance coverage, how does one know if these services are making a difference in health outcomes? The Agency of Healthcare Research and Quality (AHRQ) has developed Prevention Quality Indicators (PQIs) that focus on inpatient ambulatory care sensitive conditions (ACSCs). An ACSC is a condition that could have been avoided or prevented from becoming more severe with proper ambulatory (e.g., outpatient or preventive) care (Agency of Healthcare Research and Quality [AHRQ], 2002). ACSCs offer a way for healthcare providers, researchers, and policy makers to measure the effectiveness of primary care over time, regions, demographic categories, and providers (AHRQ, 2002). The AHRQ PQIs consist of 13 measures, including: diabetes short-term complication, perforated appendix, diabetes long-term complication, chronic obstructive pulmonary disease or asthma in older adults, hypertension, heart failure, low birth weight, dehydration, bacterial pneumonia

82 urinary tract infection, uncontrolled diabetes, asthma in younger adults, and lowerextremity amputation among patients with diabetes (AHRQ, 2002). These indicators allow identification of need levels, resources, and intervention progress monitoring (AHRQ, 2002). Ambulatory Care Sensitive Conditions and Preventive Services Literature indicates that lack of primary care is one of the major predictors for ACSC hospitalizations. A systematic review of the relationship between primary healthcare and chronic disease ACSCs found that an increase in primary healthcare resulted in a significant decrease in ACSCs, even after controlling for health status (Gibson, Segal, & McDermott, 2013). Another review found that the relationship between primary care and ACSC hospitalizations varied geographically, with areas with more access to primary care resources having fewer ACSCs hospitalizations (Rosano et al., 2013), further strengthening the relationship between the two. This relationship between primary care and ACSCs is so strong that there have even been studies that use ACSC hospitalizations a markers of primary healthcare efficacy (Caminal, Starfield, Sanchez, Casanova, & Morales, 2004). Disparities in Ambulatory Care Sensitive Conditions Both racial and ethnic minorities and low-income individuals experience poorer health and health outcomes compared to their White and higher income counterparts, including rates of ACSCs. Race and ethnicity are predictors of greater amounts of preventable hospitalizations for some ACSCs, but not all (O'Neil et al., 2010). In a study examining predictors of ACSC hospitalizations in South Carolina

83 minority, low-income, and rural populations experienced significantly higher rates ACSC hospitalizations (Shi, Samuels, Pease, Bailey, & Corley, 1999). Johnson et al. found that ACSC ED visits were related to being female, a racial/ethnic minority, older, covered by public insurance, and from a low-income neighborhood (Johnson et al., 2012). It is important to note that racial disparities for ED utilization for chronic ACSCs cannot be explained by differences in disease prevalence or severity, but rather lack of ongoing primary care (Oster & Bindman, 2003). A study found that racial disparities in ED ACSC visits persist even when controlling for gender, age, and non-ACSC admission rates (Laditka, Laditka, & Mastanduno, 2003). In sum, the majority of studies found that the racial disparities in hospitalizations and ED visits for ACSCs were due to lack of adequate and ongoing primary care, even when other factors were taken into account (O'Neil et al., 2010; Shi et al., 1999). In addition to individual characteristics, ACSC hospitalizations have been shown to be associated with certain area-level characteristics. A study using data from a public hospital system in Texas found that lower-income zip codes were associated with higher rates of ACSCs (Djojonegoro, Aday, Williams, & Ford, 2000). Another study showed that although disparities in rates of ACSC between low- and high-poverty neighborhoods decreased from 2008-2013, high-poverty neighborhoods still had ACSC rates two to four times higher than low-poverty neighborhoods in 2013 (Bocour & Tria, 2016). Finally, a statistical brief on PQI rates using data from the Healthcare Cost and Utilization Project found that in 2005-2011, rural areas had much higher rates of PQIs than less urban areas (Torio & Andrews, 2006)

84 Disparities in Preventive Care Economically disadvantaged populations have been shown to have poorer access to preventive services than high-income populations. For example, a study of the utilization of six preventive services (HIV test, smoking cessation discussion, flu shot, pneumococcal vaccination, tetanus vaccination, and zoster vaccination) using NHIS data from 2011-2012 found that those with income >200\% FPL were more likely to receive all services but HIV test (Fox & Shaw, 2014). A similar study examining blood pressure screening, cholesterol screening, colon cancer screening, diet counseling, blood sugar check, hepatitis A vaccination, hepatitis B vaccination, mammogram, and pap smear found that those with incomes >200\% FPL were significantly more likely to receive all services, except for hepatitis A vaccination (Fox & Shaw, 2015). Racial and ethnic minorities face substantial disparities in access to care and health outcomes. Non-Latino Blacks (DHHS, 2016b), Asians (DHHS, 2014a), and Latinos (DHHS, 2015a) have lower rates of health insurance and experience higher rates of morbidity and mortality than non-Latino Whites. Research has shown evidence that ACA can reduce health disparities (Chen et al., 2016). On the other hand, research has also shown that minorities often have higher rates of preventive service utilization than non-Latino Whites (Holden, Chen, & Dagher, 2015).

\end{document}
